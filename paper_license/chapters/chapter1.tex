\chapter{Specificații Funcționale}

Funcția principală a acestei aplicații este de a oferi utilizatorilor o soluție completă și ușoară pentru a închiria pe cineva care să aibă grijă de nevoile potențiale ale animalului lor de companie. Utilizatorii aplicației se încadrează în două categorii mari: „Owners” și „Walkers”. Un utilizator care are un animal de companie și are nevoie de ajutor este un „Owner”, iar, de partea opusă, un utilizator care răspunde acestei cereri este un „Walker”.

\section{Descrierea problemei}

În prezent, există o varietate de aplicații de îngrijire a animalelor de companie pe piață, care acoperă diferite aspecte, cum ar fi plimbări, programarea unei vizite la salon sau la veterinar și servicii de „sitting" a animalelor de companie. Cu toate acestea, majoritatea aplicațiilor se concentrează pe un singur aspect sau specie de animal de companie, iar utilizatorii trebuie să folosească mai multe aplicații pentru a-și gestiona sarcinile zilnice. În plus, majoritatea aplicațiilor nu oferă proprietarilor o modalitate de a găsi îngrijitori de animale de companie dacă aceștia nu au timpul necesar pentru a-și îngriji animalele de companie. 

Aplicația prezentată în această lucrare nu își propune să integreze toate aceste aplicații într-una singură, ci să ofere o soluție comodă în completarea aplicațiilor dezvoltate de diverse companii de îngrijire a animalelor care necesită deplasarea lor la un sediu fizic. În același timp, aplicația acoperă o gamă mai largă de specii de animale de companie în comparație cu alte aplicații disponibile, incluzând nu numai animale „clasice” precum câini și pisici, ci și rozătoare, păsări, reptile, pești etc. Fiecare persoană este liberă să își adauge propriul animal de companie în aplicație, indiferent de tipul de animal, iar „Walker” este liber să aleagă tipul de animal de care vrea să aibă grijă, astfel încât nimeni să nu fie lăsat în urmă.

Întrucât există o gama largă de specii și servicii disponibile în aplicație, „Walker” poate deveni dezorientat atunci când navighează și selectează serviciile. În același timp, unii oameni au fobii diferite, cum ar fi cinofobia\footnote{Cinofobia (cuvânt care provine din alăturarea a două cuvinte: latinescul canis sau grecescul kýon „câine“ și fobie) este o fobie (teamă patologică) de câini. } sau fobia de reptile. Acești oameni nu ar dori ca în aplicația lor să apară servicii care nu sunt relevante pentru ei. Din acest motiv, am decis să introducem un sistem de preferințe pentru fiecare „Walker” care ne permite să oferim o anumită prioritate pe specie de animal și serviciu. Prin urmare, plimbătorii pot alege să ofere doar plimbări pentru câini sau pisici și nu servicii veterinare pentru rozătoare. De asemenea, plimbătorii pot alege să ofere servicii de plimbare pentru câini, pisici și rozătoare, dar nu pot oferi servicii veterinare pentru nicio specie. În acest fel, aplicația devine mai ușor de utilizat și mai intuitivă pentru toți utilizatorii.


\section{Cerințe funcționale}

Principala cerință a acestei aplicații este de a oferi utilizatorilor o interfață intuitivă și ușor de utilizat, care să medieze comunicarea între „Owner” și „Walker”. În același timp, atât „Owner-ul”, cât și „Walker-ul” ar trebui să folosească aceeași aplicație, dar aceasta ar trebui să ofere capacități diferite specifice fiecăruia. În cele ce urmează, vom prezenta cerințele funcționale pentru fiecare tip de utilizator.

\subsection{Cerințe funcționale pentru „Owners”}

După conectare, prima cerință și sarcină a utilizatorului este de a adauga unul sau mai multe animale de companie in aplicație. Pentru acestea există un meniu dedicat de adăugare și modificare unde puteți alege numele, specia, rasa, data nașterii, sexul și vom putea atașa o scurtă descriere dedicata „Walker-ului”. După adăugarea unui animal de companie, acesta va apărea în lista de animale de companie a utilizatorului. Aceste informații vor fi folosite de către „Walker” pentru a decide dacă poate să se ocupe de animalul respectiv. 

După adăugarea unui animal de companie, utilizatorii pot căuta un „Walker” care să aibă grijă de animalul de companie. Pentru a face acest lucru, utilizatorul trebuie să plaseze anunțuri în aplicație. Anunțul trebui să conțină cele mai importante informații de care ar avea nevoie un „Walker”. Acestea includ profilul animalului, dată și ora la care persoană are nevoie de asistență, tipul serviciului solicitat, durata acestuia, locația la care trebuie să ajungă animalul de companie dacă este cazul și un scurt mesaj către „Walker”. Odată ce anunțul este publicat, acesta va apărea în lista de anunțuri a utilizatorului. În acest moment, anunțul va fi vizibil pentru toți „Walker-ii”. După ce unul din acestea a acceptat anunțul, „Owner-ul” poate vedea asta în aplicație în dreptul rezervării.

Pentru a crește acuratețea și a evita confuzia, locațiile clinicilor veterinare sau a saloanelor pentru animele sunt selectate folosind o hartă. În acest fel, „Walker-ul” știe exact unde să meargă pentru a ajunge la locația dorită. Același lucru este valabil și pentru adresa utilizatorului. Acesta trebuie să selecteze pe hartă o locație de unde „Walker-ul” va ridica animalul de companie.

"Owner-ul" va avea la dispoziție și un meniu în care poate vizualiza istoricul anunțurilor publicate de acesta și satisfăcute de către "Walker-i".

\subsection{Cerințe funcționale pentru „Walkers”}

În aplicație un „Walker” are două îndatoriri principale: să caute noi anunțuri care i se potrivesc și să satisfacă anunțurile pe care le-a acceptat deja.

Pentru a căuta anunțuri noi, „Walker-ul” are o listă cu toate anunțurile postate de proprietarii, care sunt relevante pentru acel „Walker”. Pentru a face lista mai ușor de navigat, această poate fi filtrată după specie și serviciu. Pentru a modifică preferințele sale legate de animale sau servicii, „Walker-ul” are la dispoziție un meniu dedicat. Acesta poate alege pentru fiecare specie de animal de companie sau serviciu un nivel de prioritate. Aceste preferințe vor fi folosite de către aplicație pentru a afișa anunțuri relevante pentru „Walker”.

După ce a găsit un anunț care i se potrivește, „Walker-ul” poate vedea detaliile anunțului pentru a-și da seama dacă dorește să îl accepte. Pe lângă detaliile introduse de utilizator, și enumerate mai sus, „Walker-ul” poate vizualiza și o harta care conține drumul de la locația sa curentă până la locația „Owner-ului” și după, dacă este cazul, până la locația unde trebuie să îl ducă pe animalul de companie. Pe lângă traseu „Walker-ul” poate vizualiza distanțele și timpul estimat pentru a ajunge la locația "Owner-ului" de la locația sa și de la locația "Owner-ului" la locația unde trebuie să îl ducă, dacă este cazul. Aceste informații sunt oferite de Google Maps API\cite{google_maps_platform} și Distance Matrix API\cite{distance_matrix}. Acestea sunt calculate pentru mersul pe jos în momentul în care  „Walker-ul” vizualizează anunțul. Timpii necesari pot varia în funcție de traficul din zona la momentul respectiv.

După ce anunțul este acceptat, „Walker-ul” poate vedea asta în aplicație în meniul special pentru acestea. În momentul în care „Walker-ul” dorește să înceapă serviciul îi va fi afișată o harta care conține locațiile la care acesta trebuie să ajungă pentru a prelua animalul de companie și locația unde trebuie să îl ducă, dacă este cazul. În același timp, el poate vedea distanțele și timpul estimat pentru a ajunge la locațiile respective. Totodată pe harta va apărea și locația acestuia în timp real. Aceste informații vor fi disponibile și pentru "Owner" dacă acesta deschide anunțul în momentul în care „Walker-ul” a început activitatea. În acest fel, "Owner-ul" poate vedea când „Walker-ul” este pe drum spre el și poate să îl aștepte în locația stabilită. După preluarea animalului de companie „Owner-ul” poate vizualiza în continuare locația  „Walker-ului” pentru a se asigura că totul decurge bine.

\section{Cerințe non-funcționale}
 
Pe lângă cerințele funcționale, aplicațiile trebuie să îndeplinească și multe cerințe nefuncționale. Acestea nu au nicio legătură cu funcționalitatea aplicației, dar sunt importante pentru a asigura o experiență plăcută atât utilizatorilor, cât și programatorilor care vor lucra la această aplicație în viitor. Aceste cerințe includ două lucruri cheie: performanță și scalabilitate. 

\newpage

\subsection{Performanță}

Una dintre caracteristicile notabile ale framework-ul Flutter este reprezentată de performanța crescută a acestuia, atât pe platforma Android cât și pe iOS. Datorită arhitecturii native Flutter și compilării AOT (Ahead-of-Time)\footnote{În programare, compilarea Ahead-of-Time (AOT) reprezintă compilarea unui limbaj de programare de nivel înalt într-un limbaj de nivel inferior înainte de execuția programului, de obicei la momentul compilării, pentru a reduce cantitatea de lucru necesară la momentul execuției.}, aplicațiile create cu aceste framework rulează eficient și fără probleme. Ca limbaj de programare Flutter folosește Dart care compilează cod nativ pe mai multe platforme, inclusiv pe cele mobile profitând de avantajele native ale dispozitivelor pe care rulează. Pentru a obține performanțe maxime, Flutter folosește un motor de randare numit Skia\footnote{https://skia.org/docs/} care este folosit și de Google Chrome. Acesta este un motor de randare 2D care oferă o performanță excelentă și o experiență de utilizare fluidă.

\subsection{Scalabilitate}

Flutter este cunoscut pentru abordarea reactivă și arhitectura bazată pe widget care facilitează dezvoltarea de aplicații scalabile. Structura widgetului permite o separare clară a responsabilităților și o reutilizare sporită a codului. Aceasta înseamnă că dezvoltatorii pot crea și gestiona cu ușurință interfețe complexe și dinamice, indiferent de dimensiune sau complexitate.

.NET, pe de altă parte, are suport nativ pentru scalabilitate și este folosit pentru a dezvolta aplicații de la desktop și web până la servicii cloud și aplicații pentru întreprinderi. Platforma .NET oferă capabilități puternice de gestionare a memoriei, concurență și distribuție care permit dezvoltatorilor să construiască și să ruleze aplicații scalabile, de performanță înaltă. În plus, framework-urile și serviciile suplimentare precum ASP.NET și Azure oferă instrumente avansate pentru scalabilitatea aplicațiilor web și cloud.

Atât Flutter, cât și .NET beneficiază de comunități active de dezvoltatori și de sprijin din partea unor jucători mari precum Google în cazul Flutter și Microsoft pentru .NET. Deci dezvoltatorii au acces la resurse, documentație și actualizări continue pentru a-i ajuta să depășească provocările de scalabilitate și să implementeze soluții eficiente.

