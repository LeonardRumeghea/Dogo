\chapter*{Concluzii} 
\addcontentsline{toc}{chapter}{Concluzii}

Animalele de companie joacă un rol semnificativ în viețile noastre, aducând bucurie, companie și beneficii pentru sănătatea noastră emoțională și fizică. La rândul lor, ele necesită îngrijire și atenție adecvate pentru a se dezvolta și a fi sănătoase. În acest scop, am dezvoltat o aplicație mobilă care oferă posibilitatea oamenilor de a se ocupa nevoile prietenilor lor necuvântători chiar si atunci timpul nu le permite acest lucru sau apar situații neprevazute. Aplicația oferă posibilitatea de a găsi o persoană disponibilă să aibă grijă de animalul de companie în lipsa proprietarului, fie că este vorba de o plimbare, o plecare de câteva zile din oraș, o vizită la veterinar sau la salonul de înfrumusețare atunci când proprietarul nu disponibil, din păcate. Aceasta aplicație nu înlocuiește grija si dragostea proprietarului, ci doar vine in ajutorul acestuia atunci când nu poate fi prezent.

Aplicația este formata din 2 componente: serverul, realizat in ASP.NET 7.0 folosind o arhitectura "Clean Arhitecture" și o aplicația mobilă realizata in Flutter, disponibilă atât pentru Android cat si pentru iOS. Aceasta din urma se folosește si de o multitudine de servicii externe pentru a oferi o mulțime de funcționalități utile si interesante pentru toți utilizatorii. Am ales să folosesc servicii externe pentru a oferi funcții scalabile si complexe, care ar fi necesitată o perioadă mai lungă de timp pentru a fi implementate. Deoarece acestea sunt oferite de companii mari precum Google, care oferă suport și actualizări constante, ceea ce asigură o aplicație robustă și sigură.

Aplicație prezentata oferă funcționalități precum: crearea unui cont, adăugarea, editarea si vizualizarea unui animal de companie, crearea unui anunț si publicarea acestuia, vizualizarea a anunțurilor publice si a celor personale sau vizualizarea unei hărții complexe care cuprinde informații precum: locații, trasee minime, locații curente, distante si timpi necesari de parcurgere.
