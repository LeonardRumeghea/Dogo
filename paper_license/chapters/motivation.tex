\chapter*{Motivație} 
\addcontentsline{toc}{chapter}{Motivație}

În societate modernă, animalele de companie joacă un rol din ce în ce mai important în viața oamenilor de zi cu zi. Cercetările recente evidențiază multiplele beneficii aduse de acestea omului, atât de natură fizică cât și psihică. Printre cele fizice se enumeră: creșterea imunității, reducerea alergiilor, îmbunătățirea sănătății cardiovasculare și activitatea fizică zilnică. Cele mai importante beneficii sunt însă de natură psihiologica, iar animalele de companie sunt o modalitate excelență de a reduce stresul cât și anxietatea provocate de rutină omului modern cât și pentru îmbunătățirea stării de spirit și creșterea a sentimentului general de bunăstare. Prin mângâierea unui câine sau a unei pisici, în organism se produc o serie de modificări fizice, prin creșterea nivelului de serotonină, numită și „hormonul fericirii”, dar și prin scăderea nivelului de cortizol, numit și „hormonul stresului”.

Cu toate acestea mulți oameni consideră că nu au sau nu pot avea destul timp la dispoziție pentru a avea grijă de un animal de companie și de nevoile acestora, cum ar fi plimbările zilnice sau vizitele periodice la salon sau veterinar. Pe lângă acestea mai apare și o problema în momentul în care persoane dorește să fie plecată mai multe zilele consecutive de acasă. Aceste probleme ar putea fi rezolvate totuși folosind o aplicație specializată pe nevoile "programate" ale animalului. Această aplicație este un intermediar dintre cele 2 persoane: una fiind deținătorul animalului de companie care nu are timpul necesar pentru a se ocupă de o activitate, sau are nevoie de ajutor, iar cealaltă este reprezentată persoană, iubitoare de animale, care dorește să se îl ajute atât de deținător cât și pe animal.

